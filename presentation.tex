% Created 2015-01-09 Fri 10:21
\documentclass[presentation, bigger]{beamer}
\usepackage[utf8]{inputenc}
\usepackage[T1]{fontenc}
\usepackage{fixltx2e}
\usepackage{graphicx}
\usepackage{longtable}
\usepackage{float}
\usepackage{wrapfig}
\usepackage{rotating}
\usepackage[normalem]{ulem}
\usepackage{amsmath}
\usepackage{textcomp}
\usepackage{marvosym}
\usepackage{wasysym}
\usepackage{amssymb}
\usepackage{capt-of}
\usepackage{hyperref}
\tolerance=1000
\usepackage{minted}
\setbeamertemplate{navigation symbols}{}
\usemintedstyle{emacs}
\usepackage[utf8x]{inputenc}
\titlegraphic{\includegraphics[width=0.4\textwidth]{images/title}}
\institute[]{Software Engineer at Basho Technologies,Inc | Founder/Organizer Papers We Love \\ @zeeshanlakhani}
\usetheme{Madrid}
\author{Zeeshan Lakhani}
\date{1-9-2015 (Codemash)}
\title{Generating Circumstances}
\begin{document}

\maketitle

\begin{frame}[label=sec-]{What I Listened To}
\includegraphics[width=\textwidth]{images/hungry.png}
\end{frame}
\begin{frame}[label=sec-]{\#TDDDEADYO}
\includegraphics[width=\textwidth]{images/tdd.png}\footnote{\url{http://bit.ly/1ALmSGC}}
\end{frame}
\begin{frame}[label=sec-]{In Papers}
\begin{center}
 \includegraphics[scale=0.25]{images/papers}
\end{center}
\end{frame}
\begin{frame}[label=sec-]{In the Year 2000\ldots{}}
\begin{quote}
We have designed a simple domain-specific language of testable specifications
which the tester uses to define expected properties of the functions under test.
\end{quote}

\begin{quote}
We have chosen to put distribution under the human tester's control, by
defining a test data generation language…
\end{quote}

\begin{quote}
We have taken two relatively old ideas, namely specifications as oracles and
random testing, and found ways to make them easily available …
\end{quote}
\end{frame}
\begin{frame}[label=sec-]{Don't Write Tests\footnote{John Hughes - \url{http://bit.ly/1rDOGr3}}}
\begin{itemize}
\item One Feature - O(n)
\item Pairs of Features - O(n2) - quadratic
\item Triples of Features - O(n3) - cubic
\end{itemize}
\end{frame}
\begin{frame}[label=sec-]{Thinking in Specifications\footnote{the @seancribbs - \url{http://bit.ly/1wZm0I6}}}
\begin{itemize}
\item A “roundtrip”, e.g encode/decode?
\item An existing implementation with similar behavior
\item A relationship between inputs/outputs?
\item A set of client interactions with a platform
\end{itemize}
\end{frame}
\begin{frame}[label=sec-]{Sample Properties\footnotemark[3]{}}
\begin{itemize}
\item Reversing a list twice should equal the original list.
\item Reversing and sorting a list should preserve its length.
\item Popping an element from a queue should reduce its size by one
\end{itemize}
\end{frame}
\begin{frame}[label=sec-]{Defining Truth?\footnote{Notes on Erlang QC - \url{http://bit.ly/14CmDBF}}}
\begin{itemize}
\item Invariant
\end{itemize}
\begin{block}{?SOMETIMES(N,Prop)}
A property which tests Prop repeatedly N times, failing only if all of the
tests fail. In other words, the property passes if Prop sometimes passes. This
is used in situations where test outcomes are non-deterministic, to search for
test cases that consistently fail\ldots{}
\end{block}

\begin{block}{fails(Prop::property()) -> property()}
A property which succeeds when its argument fails. Sometimes it is useful to
write down properties which do not hold (even though one might expect them to).
This can help prevent misconceptions.
\end{block}
\end{frame}

\begin{frame}[label=sec-]{The Only Sure Thing in Computer Science}
\begin{itemize}
\item Everything is a tradeoff.\footnote{\url{http://bit.ly/1xP7uIg}}
\begin{itemize}
\item QC Tradeoffs: Duration | Assertion Power
\end{itemize}
\end{itemize}
\end{frame}
\begin{frame}[label=sec-]{A Personal Story}
\begin{itemize}
\item ClojureWest 2014

\item John Hugues / Reid Draper
\end{itemize}

\begin{center}
 \includegraphics[scale=0.25]{images/story}
\end{center}
\end{frame}
\begin{frame}[fragile,shrink,label=sec-]{\#BEZERKER}
 \begin{minted}[]{clojure}
;; Release-Base and Release-Link are at the top of this file. This is
;; so Media-Link can correctly depend on Release-Link.
(def Release-New
  (assoc Release-Base
    ;; TODO: This should probably be optional, but I really want csync to send it
    ;; so leaving it as required just for now.
    :slug c/Slug
    :name c/NonEmptyString
    :artists (s/both [c/Simple-Account-Link] c/NonEmptyCollection)
    ;; TODO: This should probably be optional, but I really want csync to send it
    ;; so leaving it as required just for now.
    :created-by c/Simple-Account-Link
    (s/optional-key :copyright) (s/maybe c/Localized-Map)
    :media (s/both [(s/either c/Simple-Track-Link c/Simple-Video-Link)]
                   c/NonEmptyCollection)
    (s/optional-key :purchase) c/Simple-Link
    (s/optional-key :label) c/Simple-Account-Link
    (s/optional-key :images) c/Images
    (s/optional-key :description) c/Localized-Map
    (s/optional-key :created-with) c/Simple-Account-Link
    (s/optional-key :uploaded-with) c/Simple-Account-Link
    (s/optional-key :hearted) s/Bool
    (s/optional-key :pro-id) s/Str
    (s/optional-key :pro-slug) s/Str))

(def Release-Existing
  (assoc Release-Base
    :name c/NonEmptyString
    :slug c/Slug
    :prior-slugs [c/Slug]
    :created-at sc/ISO-Date-Time
    :created-by a/Account-Link
    :artists [a/Account-Link]
    :label (s/maybe a/Account-Link)
    :images c/Images
    :hearted s/Bool
    :copyright (s/maybe c/Localized-Map)
    :description c/Localized-Map
    :media [(s/either Track-Link Video-Link)]
    :sharing c/Simple-Link
    :total-hearts s/Int
    ;; :total-plays s/Int
    :created-with (s/maybe a/Account-Link)
    :uploaded-with (s/maybe a/Account-Link)
    :purchase (s/maybe c/Simple-Link)
    :pro-id (s/maybe s/Str)
    :pro-slug (s/maybe s/Str)
    ;; :license License-Link
    :hearted-by c/Simple-Link
    ;; :listened-to c/Simple-Link
    ))

(def Playlist-Base
  {:name c/NonEmptyString
   (s/optional-key :tags) c/Tags
   (s/optional-key :duration-seconds) (s/maybe s/Int)})

(def Playlist-Link
  (assoc Playlist-Base
    :url c/URL
    :created-by a/Account-Link
    :images c/Images
    :sharing c/Simple-Link
    :pro-id  (s/maybe s/Str)
    :pro-slug (s/maybe s/Str)
    ;; :total-hearts s/Int
    ;; :total-plays s/Int
    ))
\end{minted}
\end{frame}
\begin{frame}[fragile,shrink,label=sec-]{}
 \begin{minted}[]{clojure}
(def Mix-Existing
  (assoc Mix-Base
    :slug c/Slug
    :prior-slugs [c/Slug]
    :created-at sc/ISO-Date-Time
    :created-by a/Account-Link
    :artists [a/Account-Link]
    :release (s/maybe Release-Link)
    :label (s/maybe a/Account-Link)
    :images c/Images
    :hearted s/Bool
    :copyright (s/maybe c/Localized-Map)
    ;; :license c/Simple-Link
    :description c/Localized-Map
    :source-tracks [{:track Track-Link
                     :start-time-seconds s/Int}]
    :sharing c/Simple-Link
    :total-hearts s/Int
    ;; :total-plays s/Int
    :purchase (s/maybe c/Simple-Link)
    :created-with (s/maybe a/Account-Link)
    :uploaded-with (s/maybe a/Account-Link)
    :recorded-date (s/maybe sc/ISO-Date-Time)
    :hearted-by c/Simple-Link
    ;; :listened-to c/Simple-Link
    ))
\end{minted}
\end{frame}
\begin{frame}[label=sec-]{quickcheck in the wild}
\begin{itemize}
\item \href{https://github.com/clojure/test.check}{test.check (clojure)}
\item \href{https://hackage.haskell.org/package/QuickCheck}{Quickcheck - Haskell}
\item \href{http://www.quviq.com/products/erlang-quickcheck/}{Erlang Quickcheck (from QuviQ)}
\item \href{http://www.scalacheck.org/}{ScalaCheck}
\item \href{http://jsverify.github.io/}{JSVerify}
\item \href{https://github.com/fsharp/FsCheck}{FsCheck (for .NET)}
\item \ldots{}
\end{itemize}
\end{frame}
\begin{frame}[fragile,label=sec-]{Specification - The Transpose of a Transposed Matrix is the Original Matrix}
 \begin{block}{(A\^{}T)\^{}T = A}
\begin{minted}[]{clojure}
(def transpose-of-transpose-prop
  (prop/for-all [m matrix-gen]
  (= m (transpose (transpose m)))))

(quick-check 50 transpose-of-transpose-prop)

;; Results:
{:result true, :num-tests 50, :seed 1405444353915}
\end{minted}
\end{block}
\end{frame}
\begin{frame}[fragile,shrink,label=sec-]{}
 \begin{minted}[]{clojure}
(def matrix
  [[1 2]
   [3 4]])

(transpose matrix)

;; Results:
[[1 3]
 [2 4]]
\end{minted}
\end{frame}
\begin{frame}[fragile,shrink,label=sec-]{}
 \begin{minted}[]{clojure}
(def matrix-gen
  (gen/such-that
   not-empty
   (gen/vector
    (gen/tuple gen/int gen/int gen/int))))

(gen/sample matrix-gen 10)

;; Results:
([[0 0 1]]
 [[0 -1 0]]
 [[0 2 1] [0 1 -2]]
 [[1 0 1] [0 1 -3] [0 4 2] [2 -3 4]]
 [[-3 5 -4] [3 -3 -2] [-2 3 -2] [-3 -5 -3] [0 -3 -1]]
 [[2 -5 -3] [-4 -3 5] [-4 -3 -4] [-4 4 3]]
 [[-4 4 5] [-4 2 0] [5 -6 0] [2 -3 5] [-6 -3 -5]]
 [[-5 2 -3] [-2 -2 5]]
 [[-1 3 -5] [5 -3 -2] [7 4 -7] [7 -3 -2] [1 -3 8]]
 [[-5 -3 -3] [2 7 -3]])
\end{minted}
\end{frame}
\begin{frame}[fragile,label=sec-]{gen/fmap}
 \alert{gen/fmap} allows us to create a new generator by applying a function to the
values generated by another generator\footnote{test.check docs - \url{http://bit.ly/1xWxQ9R}}

\begin{minted}[]{clojure}
(def gen1
  (gen/fmap (fn [n] (* n 2)) gen/nat))

(gen/sample gen1)

;; Results
(0 2 4 2 0 10 10 6 8 8)
\end{minted}
\end{frame}
\begin{frame}[fragile,label=sec-]{gen/bind}
 \alert{gen/bind} allows us to create a new generator based on the value of a
previously created generator\footnotemark[6]{}

\begin{itemize}
\item Generator a -> (a -> Generator b) -> Generator b
\end{itemize}

\begin{minted}[]{clojure}
(def gen2
  (gen/bind gen1 (fn [v]
                 (gen/hash-map :codemash
                               (gen/return v)))
(gen/sample gen2)

;; Results
({:codemash 0} {:codemash 2} {:codemash 0}
 {:codemash 4} {:codemash 6} {:codemash 10}
 {:codemash 2} {:codemash 14} {:codemash 12}
 {:codemash 16})
\end{minted}
\end{frame}
\begin{frame}[fragile,label=sec-]{seed}
 Running Against the Same Set of Test Cases

\begin{minted}[]{clojure}
(def prop-correct
  (prop/for-all [v gen1]
                (> 100 v)))

(tc/quick-check 100 prop-correct :seed 1420668333773)
\end{minted}
\end{frame}
\begin{frame}[fragile,label=sec-]{Shrinking\footnote{video - \url{http://bit.ly/1zYwPwa}}}
 Shrink trees are \alert{lazily} generated for each generated result value

\begin{center}
 \includegraphics[scale=0.3]{images/flattened_joined_shrink}
\end{center}

\rule{\linewidth}{0.5pt}

Remember our property \alert{(> 100 v)}?

\begin{minted}[]{clojure}
{:result false, :seed 1420668333773, :failing-size 63,
 :num-tests 64, fail [108],
 :shrunk {:total-nodes-visited 17, :depth 2,
          :result false,
 :smallest [100]}}
\end{minted}
\end{frame}
\begin{frame}[label=sec-]{schema->gen\footnote{\url{http://bit.ly/1tSakMP}} for reasons}
\begin{itemize}
\item Turn types, generics, schemas into generated data.
\item My use-case: Prismatic Schema\footnote{\url{http://bit.ly/1BTaPW4}}\textsuperscript{,}\,\footnote{another example - Herbert - \url{http://bit.ly/1IxJs5V}}
\item Work with Properties Testing API Workflow
\item Regression Testing Out of the Box
\end{itemize}
\end{frame}
\begin{frame}[fragile,shrink,label=sec-]{}
 \begin{minted}[]{clojure}
  (def s-vector
    [(s/one s/Bool "first")
     (s/one s/Num "second")
     (s/one #"[a-z0-9]" "third")
     (s/optional s/Keyword "maybe")
     s/Int])

  ;; (true 3.0 "r"
  ;; :_1:r98l:Y!:npG-*:ZLyx4*+?:+I7:yO8577B5D:392_:!1-+2:8-aMu7
  ;; 1 7 3 -4)

  (def s-hashmap-with-hashmap
    {:foo s/Int
     :baz s/Str
     :bar {:foo s/Int}
     :far {(s/optional-key :bah) s/Bool}
     s/Keyword s/Num})

;; {:foo 0,
;;  :baz "%?I\"",
;;  :bar {:foo 9},
;;  :far {:bah false},
;;  :j 1.0,
;;  :O*37L:?K7+43:M?!?U_DIl*:GS90Ky**11 2.0}

  (s/check s-hashmap-with-hashmap datum)
\end{minted}
\end{frame}
\begin{frame}[fragile,label=sec-]{A more realistic shrink}
 \begin{itemize}
\item Trying to Model Recursive Data Types\footnote{\url{http://bit.ly/1sc221b}}
\end{itemize}

\begin{minted}[]{clojure}
{:foo -1.53125, :baz {:foo -3.0,
 :baz {:baz {:baz {:baz {:baz {:foo -1.1891892}}}}}}}

;; :smallest [{:foo 0.0, :baz {:foo 0.0,
;;             :baz {:baz {:foo 0.0}}}}]
\end{minted}
\end{frame}

\begin{frame}[fragile,shrink,label=sec-]{}
 \alert{Multiple Dispatch}

\begin{minted}[]{clojure}
(defmethod schema->gen* schema.core.One
  [e]
  (schema->gen (:schema e)))

(defmethod schema->gen* schema.core.RequiredKey
  [e]
  (gen/return (:k e)))

(defmethod schema->gen* schema.core.OptionalKey
  [e]
  (gen/return (:k e)))

(defmethod schema->gen* schema.core.Maybe
  [e]
  (gen/one-of
   [(gen/return nil)
    (schema->gen (:schema e))]))
\end{minted}
\end{frame}

\begin{frame}[fragile,shrink,label=sec-]{}
 \alert{Composing Generators}

\begin{minted}[]{clojure}
(defmethod schema->gen* clojure.lang.Sequential
  [e]
  (let [[ones [repeated]]
        (split-with #(instance? schema.core.One %) e)
        [required optional]
        (split-with (comp not :optional?) ones)]
    (g/apply-by
     (partial apply concat)
     (g/one-of
      (apply gen/tuple (map schema->gen required))
      (g/apply-by
       (partial apply concat)
       (apply gen/tuple
              (map schema->gen
                   (concat required optional)))
       (if repeated
         (gen/vector (schema->gen repeated))
         (gen/return [])))))))
\end{minted}
\end{frame}
\begin{frame}[label=sec-]{Changing Gears}
\begin{center}
 \includegraphics[scale=0.3]{images/merkletree}
\end{center}
\end{frame}
\begin{frame}[label=sec-]{eqc\_statem}
Model State Transitions (as a FSM) -> Assert Against Implementation\footnote{@jtuple - \url{http://bit.ly/1tOfzaR}}

\begin{center}
 \includegraphics[scale=0.25]{images/eqc_statem}
\end{center}
\end{frame}
\begin{frame}[label=sec-]{Side Effects}
\begin{itemize}
\item Requires knowledge about the \alert{context} and its possible
\alert{histories}\footnote{\url{http://bit.ly/14CyorP}}
\item \alert{Symbolic values} are generated during test \alert{generation}
    and \alert{dynamic values} are computed during test \alert{execution}
\begin{itemize}
\item dynamic state is computed at runtime
\end{itemize}
\item \alert{next\_state} callback operates during both test \alert{generation}
     and test \alert{execution}
\end{itemize}
\end{frame}
\begin{frame}[fragile,shrink,label=sec-]{}
 \begin{minted}[]{erlang}
%% ====================================================================
%% Notes
%% ====================================================================

%% [1] Earle, Clara Benac, and Lars-Ake Fredlund."Testing Java with QuickCheck."

%% This is a very basic eqc_statem test that I've updated a bit, dealing with
%% adding|cons'ing to a list and making sure those added values are members of a
%% list.

%% ====================================================================
%% Code
%% ====================================================================

-module(stateful_sm).
-compile(export_all).

-include_lib("eqc/include/eqc.hrl").
-include_lib("eqc/include/eqc_statem.hrl").

setup() ->
    io:format("Setup Components If Need Be.~n"),
    ok.

cleanup() ->
    io:format("TearDown Components if Need Be.~n"),
    ok.

test() ->
    test(100).

test(N) ->
    setup(),
    try eqc:quickcheck(numtests(N, prop_codemash()))
    after
        cleanup()
    end.
\end{minted}
\end{frame}
\begin{frame}[fragile,shrink,label=sec-]{}
 \begin{minted}[]{erlang}
%% Initialize State
initial_state() -> [].

%% ------ Grouped operator: add
%% @doc add_command - Command generator
-spec add_command(S :: eqc_statem:symbolic_state()) ->
        eqc_gen:gen(eqc_statem:call()).
add_command(S) ->
    {call, ?MODULE, add, [S, nat()]}.

%% @doc add_pre - Precondition for add
-spec add_pre(S :: eqc_statem:symbolic_state(),
              Args :: [term()]) -> boolean().
add_pre(S, _Args) ->
    S /= undefined.

%% @doc add_next - Next state function
-spec add_next(S :: eqc_statem:symbolic_state(),
               V :: eqc_statem:var(),
               Args :: [term()]) -> eqc_statem:symbolic_state().
add_next(S, _Value, [_, N]) ->
    [N|S].

%% @doc add_post - Postcondition for add
-spec add_post(S :: eqc_statem:dynamic_state(),
               Args :: [term()], R :: term()) -> true | term().
add_post(S, [_, N], Res) ->
    [N|S] =:= Res.

%% @doc - Perform add action
-spec add(list(), non_neg_integer()) -> list().
add(AList, N) ->
    [N|AList].
\end{minted}
\end{frame}
\begin{frame}[fragile,shrink,label=sec-]{}
 \begin{minted}[]{erlang}
%% ------ Grouped operator: is_member
%% @doc is_member_command - Command generator
-spec is_member_command(S :: eqc_statem:symbolic_state()) ->
        eqc_gen:gen(eqc_statem:call()).
is_member_command(S) ->
    {call, ?MODULE, is_member, [S, nat()]}.

%% @doc is_member_pre - Precondition for is_member
-spec is_member_pre(S :: eqc_statem:symbolic_state(),
                    Args :: [term()]) -> boolean().
is_member_pre(S, _Args) ->
    S /= undefined.

%% @doc is_member_next - Next state function
-spec is_member_next(S :: eqc_statem:symbolic_state(),
                     V :: eqc_statem:var(),
                     Args :: [term()]) -> eqc_statem:symbolic_state().
is_member_next(S, _Value, _Args) ->
    S.

%% @doc is_member_post - Postcondition for is_member
-spec is_member_post(S :: eqc_statem:dynamic_state(),
                     Args :: [term()], R :: term()) -> true | term().
is_member_post(S, [_, N], Res) ->
    Res == lists:member(N, S).

%% @doc - Perform is_member action
-spec is_member(list(), non_neg_integer()) -> boolean().
is_member(S, N) ->
    lists:member(N, S).
\end{minted}
\end{frame}
\begin{frame}[fragile,shrink,label=sec-]{}
 \begin{minted}[]{erlang}
%% ====================================================================
%% Invariants
%% ====================================================================

-spec invariant(eqc_statem:dynamic_state()) -> boolean().
invariant(S) when length(S) >= 0 ->
    true;
invariant(S) when length(S) > 0 ->
    FirstNum = hd(S),
    is_number(FirstNum) andalso FirstNum >= 0;
invariant(_) ->
    false.

%% Property Test
prop_codemash() ->
    ?FORALL(Cmds,commands(?MODULE),
            aggregate(command_names(Cmds),
                      begin
                      {H, S, Res} = run_commands(?MODULE, Cmds),
                      pretty_commands(?MODULE,
                                      Cmds,
                                      {H, S, Res},
                                      Res =:= ok)
                      end)).
\end{minted}
\end{frame}
\begin{frame}[fragile,shrink,label=sec-]{}
 \begin{minted}[]{erlang}
stateful_sm:test().

%% Results:

%% Licence for Basho reserved until
%% {{2015,1,8},{16,15,57}}
%% .........................
%% .........................
%% .........................
%% .........................
%% OK, passed 100 tests

%% 51.6% {stateful_sm,is_member,2}
%% 48.4% {stateful_sm,add,2}
%% true
\end{minted}
\end{frame}
\begin{frame}[fragile,shrink,label=sec-]{WIP: Modeling KV-YZ HashTree in Riak}
 \begin{minted}[]{erlang}
%% ------ Grouped operator: start_yz_tree
%% @doc start_yz_tree_command - Command generator
-spec start_yz_tree_command(S :: eqc_statem:symbolic_state()) ->
        eqc_gen:gen(eqc_statem:call()).
start_yz_tree_command(_S) ->
    {call, ?MODULE, start_yz_tree, []}.

%% @doc start_yz_tree_pre - Precondition for generation
-spec start_yz_tree_pre(S :: eqc_statem:symbolic_state()) -> boolean().
start_yz_tree_pre(S) ->
    S#state.yz_idx_tree == undefined.

%% --------------------------------------------------------------------

-spec insert_kv_tree(sync|async, obj(), {ok, tree()}) -> ok.
insert_kv_tree(Method, RObj, {ok, TreePid}) ->
    {Bucket, Key} = eqc_util:get_bkey_from_object(RObj),
    Items = [{void, {Bucket, Key}, RObj}],
    case Method of
        sync ->
            riak_kv_index_hashtree:insert(Items, [], TreePid);
        async ->
            riak_kv_index_hashtree:async_insert(Items, [], TreePid)
    end.
\end{minted}
\end{frame}
\begin{frame}[label=sec-]{We Talking About Tests? Tests?}
\begin{itemize}
\item Automating Selenium Actions\footnote{Kemerling - Pivotal Tracker -  \url{http://bit.ly/1w5sXHk}}
\item Generating Models
\item Sample Data
\item Investigation
\item Assertions
\end{itemize}
\end{frame}
\begin{frame}[label=sec-]{Going Further}
\begin{itemize}
\item Formal Specifications - Thinking for Programmers\footnote{\url{http://bit.ly/1w5emM2}}
\item Molly - Peter Alvaro\footnote{\url{http://bit.ly/1obnZLJ}}
\begin{itemize}
\item A system for automatically detecting errors in a
program with a correctness spec, the program, and
malevolent sentience as input
\end{itemize}
\end{itemize}
\end{frame}
% Emacs 25.0.50.1 (Org mode 8.3beta)
\end{document}
